Bayesian Optimization is a optimization algorithm that uses a probabilistic model to approximate the objective function \textit{(surrogate model)} and uses this model to guide the search for the optimal solution. The general procedure can be described as follows:
\begin{enumerate}
    \item choose surrogata function that approximate the real objective function (prior)
    \item repeat until stopping condition
          \begin{enumerate}
              \item given a number of observations (computed from the real objective function), update the surrogate function (posterior distribution). Default choice is usually a Gaussian Process
              \item optimize a cheap \textit{acquisition function / utility function} based on the posterior distribution to find the new point to sample. Also has the responsability of balancing exploration and exploitation
          \end{enumerate}
\end{enumerate}