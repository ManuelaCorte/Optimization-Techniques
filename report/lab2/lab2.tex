\subsection{DIRECT}
Partitioning algorithm that recursively subdivides the feasible region into smaller hyperrectangles. The algorithm is based on the principle of divide and conquer. The algorithm is as follows:
\begin{enumerate}
    \item Divide the feasible region into smaller hyperrectangles.
    \item Evaluate the function at the center of each hyperrectangle. The division is performed so that the region with the best function value is given the largest space.
    \item a set of potentially optimal hyperrectangles is identified and further divided.
\end{enumerate}

\subsection{Basin Hopping}
Basing Hopping is a form of Iterated Local Search with a different starting point each time. It loops through the following steps:
\begin{enumerate}
    \item Hopping: perturbation of the current solution ("jump" to new parts of the search space)
    \item Local Search: perturbed solution is optimized using a local search method
    \item Acceptance: the new solution is accepted or rejected based on an acceptance criterion. This criterion can be defined in different ways; here is defined based on the Metropolis criterion.
\end{enumerate}
% The "temperature" parameter for the acceptance or rejection criterion.
% Higher "temperatures" mean that larger jumps in function value will be
% accepted.  For best results T should be comparable to the
% separation (in function value) between local minima.

% This is a crucial parameter in `basinhopping` and
% depends on the problem being solved. The step is chosen uniformly in the
% region from x0-stepsize to x0+stepsize, in each dimension. Ideally, it
% should be comparable to the typical separation (in argument values) between
% local minima of the function being optimized. `basinhopping` will, by
% default, adjust `stepsize` to find an optimal value, but this may take
% many iterations. You will get quicker results if you set a sensible
% initial value for ``stepsize``.